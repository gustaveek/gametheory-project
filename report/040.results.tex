% BEGIN 040.results.tex. Simulation results and discussion.
\section{Results and discussion}\label{sec:results}
% Shortcomings in the MATLAB program
The \matlab\ program described in section \ref{sec:implem} has different shortcomings, mostly in performence, that makes it hard to carry out numerous demanding examinations of the game model. Nevertheless some investigations were carried out and the results are presented in this section.\mypar

During simulation fitness proportional selection were used, and the parameters were set according to Table \ref{tab:results:parameters}


\begin{table}[!hbtp]
  \footnotesize
  \centering
  \label{tab:results:parameters}
  \caption{Parameter settings during simulations. Toal number of simulations: 35}

  a) Static parameters \red{Change to exponential form}\myhalfpar

  \begin{tabular}{cc}
    \begin{tabular}{ll@{$~=~$}l}
      \hline
      Antal individer        & $\mu$                      & $100$  \\
      Antal föräldrar        & \texttt{nPar}              & variating \\
      Selection pressure     & \texttt{selectionPressure} & 0.1    \\
      Number generations     & \texttt{nGenerations}      & 10000  \\
      Mut. float length      & \texttt{mutFloatLength}    & 0.05   \\
      Payoff for \textsc{cc} & \texttt{payoffCC}          & 3      \\
      Payoff for \textsc{dc} & \texttt{payoffDC}          & 5      \\
      \hline
    \end{tabular}
    &
    \begin{tabular}{ll@{$~=~$}l}
      \hline
      Payoff for \textsc{cd} & \texttt{payoffCD}          & 0         \\
      Payoff for \textsc{dd} & \texttt{payoffDD}          & 1         \\
      Prob. for switch       & \texttt{pMutSwitch}        & variating \\
      Prob. for split        & \texttt{pMutSplit}         & -''-      \\
      Prob. for duplicate    & \texttt{pMutDuplicate}     & -''-      \\
      Prob. for float mut.   & \texttt{pMutFloat}         & -''-      \\
      Prob. for error        & \texttt{pError}            & -''-      \\
      \hline
    \end{tabular}
  \end{tabular}\mypar

  b) Swept parameters\myhalfpar

  \newcommand \temp {$5\cdot 10^{-4}$}
  \begin{tabular}{*{8}{l}}
    Simulation    & 1--5               & 6--10    & 11--15    & 16--20   & 21--25   & 26--30   & 31--35   \\
    \hline
    \texttt{pMut} & $\mathbf{10^{-4}}$ & \temp       & $\mathbf{10^{-3}}$ & \temp       & \temp       & \temp       & \temp       \\
    \texttt{nPar} & $0.2\,\mu$          & $0.2\,\mu$ & $0.2\,\mu$  & $\mathbf{0.1\,\mu}$ & $0.2\,\mu$ & $\mathbf{0.4\,\mu}$ & $0.2\,\mu$ \\
    \texttt{pErr} & $0.01$             & $0.01$   & $0.01$    & $0.01$   & $0.01$   & $0.01$   & $\mathbf{0}$ 
  \end{tabular}
\end{table}

Alla populationer startar med $25 \%$ 00,01,10,11 strategier.\mypar % Vad betyder egentligen detta?

\subsection{Simulations and results}
% Interpretation of the figures
The following figures shows some results. This is the way to interpret the figures. The lower part of shows the distribution of game outcomes, and the upper part shows what strategies that occurs and to what extent they dominate the population.\mypar

The first thing investigated was the mutation rate $p_{\textrm{mut}}$. In the experiments the mutation parameters were set as $p_{\textrm{split}} = p_{\textrm{dup}} = 0.1\,p_{\textrm{mut}}$, and$ p_{\textrm{float}} = p_{\textrm{switch}} = p_{\textrm{mu}}$. It appears as if the mutation rate is low, $p_{\textrm{mut}}=0.01$ the almost pure strategy of \textco{all-cooperate} becomes dominant, but for $p_{\textrm{ut}}=0.1$ the opposite \textco{all-defect} becomes the dominant strategy.\mypar

Resultat: Inga strategier verkar osårbara men många av de som ligger nära de diskreta varianterna är
vanliga bland de dominerande strategierna. få körningar där strategier dominerar med t.ex 0.25 eller
0.3 I en bit (nr8 nr 28 nr 22, nr 25) kanske kan de vara stabilare då det är svårt att mutera dem till en
elak strategi? Skulle behövt köra riktigt långa körningar för att se om det stabiliserar sig bättre, men
förmodligen kommer det väl inte bli mer än perioder av stabilitet. Intressant med perioderna av $25\,\%$
cc, $25\,\%$ cd, $25\,\%$ dc, $25\,\%$ dd. Har inte kollat nu vad det är för sammansättning av strategier som ger
dessa utfall men man kan ju tycka att det är besynnerligt att de kan vara stabila under flera
generationer då detta betyder att det finns en grupp som utnyttjar en annan. Men det kanske inte är
samma grupp som utnyttjar varje generation I och för sig. I de flesta övergångar så går man direkt
från en dominant strategi till en annan men ibland så sker det med ett mellanliggande strategi som
språngbräda (exempel 10, 17, 19) (lite löst definierat hur länge en mellanliggande strategi får leva och
ändå kallas språngbräda).

However, what can actually be said is that there is one dominant strategy.\mypar

\paragraph{Standard simulations 6--10 och 21--25}
Av standardkörningarna 6--10 och 21--25 så är den strategi med längst liv ser ut som synes i Tablell \ref{tab:results:dominant:6}
av detta kan man ju notera att 5 stycken är ganska snarlika en 0,1 strategi. Att de två som överlevt längst är så udda strategier som 0.2453, 0.99967 och 0.1715,1 de två längsta defective strategierna är båda 0 strategier.\mypar

\begin{figure}[htbp]
\centering
\includegraphics[width=\linewidth]{pdf/standardRun22.pdf}
\caption{Illustration of a standard simulation result}
\end{figure}

\begin{figure}[htbp]
\centering
\includegraphics[angle=90, width=\linewidth]{pdf/standardRun23.pdf}
\caption{Illustration of a standard simulation result}
\end{figure}

\begin{figure}[htbp]
\centering
\includegraphics[angle=90, width=\linewidth]{pdf/standardRun25.pdf}
\caption{Illustration of a standard simulation result}
\end{figure}

väldigt blandade resultat. Det finns både körningar där de dominerande strategierna är många och byts
snabbt och andra där enstaka strategier dominerar under långa perioder. Både omgångar där DD och
CC växlar mycket, omgångar där DD dominerar överlag och andra omgångar där CC dominerar
överlag. Antal strategier I snitt 132, högst 152 lägst 106\mypar

\begin{table}[!hbtp]
  \footnotesize
  \centering
  \label{tab:results:dominant:6}
  \caption{Dominant strategies in the standard simulations, 6--10 and
    21--25. The table show those strategies that were dominant during
    the longest period in the simulations. Note that the outcome of
    simulation 7 was hard to determine.}
  \begin{tabular}{lllc}
    Simu-  & dominating & duration  & dominating \\
    lation & strategy   & (approx)  & at end \\
    \hline
    6  & 0.0110         & 4000 & x \\          
    7  & ---            & --- &            \\
    8  & 0.12, 1        & 5000 &            \\ 
    9  & 0.0            & 6000 &            \\ 
    10 & 0.09,1         & 4000 &            \\ 
    21 & 0.0578,1       & 4500 & x \\          
    22 & 0.2453, 9.9967 & 8000 & x \\          
    23 & 0.05, 1        & 3500 &            \\ 
    24 & 0.0580, 0.9991 & 3500 & x \\          
    25 & 0.1715, 1      & 7000 & x \\          
    \hline
  \end{tabular}
\end{table}

\red{One figure here!}

\paragraph{The other simulations} Jag kan nog säga varför vi ska avfärda 1:5 och 31:35 men de andra är inte helt uppenbara. Kanske
snarare handlar om att se om man får liknande resultat trots att man ändrar på parametrar. Dock får vi
ganska olika resultat. Kanske kan man se 0 som en trivial lösning. Den är ju svår att rå på. I
nedanstående spalt har jag inte kollat om de är dominerande vid 10000 och därför kan en del
potentiellt överleva längre. Där jag inte skrivit I för att jag inte vet går det givetvis att kolla upp vilka
de är men jag orkte inte det nu.\mypar

Table \ref{tab:results:dominant:11} shows dominant strategies in non standard simulations.

\begin{table}[!hbtp]
  \footnotesize
  \centering
  \label{tab:results:dominant:11}
  \caption{Dominant strategies in the \textit{non} standard simulations, 11--20 and
    26--30}
  \begin{tabular}{lllc}
    Simu-  & dominating & duration  & dominating \\
    lation & strategy   & (approx)  & at end \\
    \hline
11  &  0.19,1              & 6000      \\          
12  &  0                   & 7000               \\ 
13  &  samarbetande        & 5000               \\ 
14  &  0                   & 7000               \\ 
15  &  0                   & 3000               \\ 
16  &  0                   & 4000      \\          
17  &  (0.05,0),(0.05,1)   & 2000,2000 \\          
18  &  0.06,1              & 3000               \\ 
19  &  0.06,1              & 6000      \\          
20  &  0.06,1              & 4000      \\          
26  & 0.1560, 0.9973       & 5000\\          
27  & 0.8910,1             & 6000\\          
28  & inga långvariga strat&              \\ 
29  & 0                    & 6000\\          
30  & samarbetande         & 3500\\          
    \hline
  \end{tabular}
\end{table}

\paragraph{Simulation 1--5, ändra pMut till $1\cdot 10^{-4}$}
verkar ha för låg grad av mutationer. Minst antal strategier uppkommer och det verkar vara svårt att ta över populationen.\mypar

Generellt kännetecknas körningarna av att relativt få strategier utvecklas under 10000 generationer.
Vilket kan förklaras av att det är den lägsta mutationsgraden av alla körningar. Enstaka strategier
dominerar långa tidsförlopp. 4 av 5 körningar dominerar DD vid 10 000 och har ett relativt starkt
övertag under hela omgången.\mypar

\red{Should we discard these and show no figures?}

\paragraph{Simulation 11--15, pMut $1\cdot 10^{-3}$}
antal strategier I snitt 175, flest 242, minst 130. vilket som väntat är mer än standardkörningen. Trots
detta förekommer strategier som är dominerande under långa tidsperioder. Men också matcher där det
skiftar fort. Två matcher domineras av CC och två av DD en är väldigt turbulent men slutar I stabilt
DD.

\paragraph{Simulation 16--20, nPar=0.1}
strategier I snitt: 87. som mest 118 som minst 70. Färre föräldrar ger färre antal strategier. Långa
perioder av både CC och DD. Men också en körning där $25\,\%$ av var är stabil under en längre period.

\paragraph{Simulation 26--30, nPar=0.4}
flest strategier av alla omgångar. I snitt ?? vilket syns I de svajiga perioderns om kännetecknas av
snabba byten och många olika strategier. Det finns dock en del perioder som kan ses som rättså
stabila. Tre matcher domineras av cc och tre dominerar I alla fall andra halvlek av dd.

\paragraph{Simulation 31--35, pErr=0}
väldigt svårt att bryta igenom dominerande strategi. Gör också att matcherna till stor del bara slutar I
att alla defektar eller att alla samarbetar. Endast kortare perioder av cd, dc eller t.ex $80\,\%$ cc
förekommer. Lite färre strategier I snitt än standardkörningen.modellen fungerar inte alls lika bra med
noll fel men jag kommer inte ihåg exakt vad det var vi skulle leta efter för att bevisa det.


\subsection{Improvements and alterternatives}
For to start with, it appeared to be time consuming to evaluate all individuals against each other.

\red{If the population dynamics was implemented as a dynamical system a'la Lotka-Volterra, then the discretion described in section \ref{sec:implem} could be avoided.}

\subsection{Further work}
(altering the parameters in the payoff matrix would be a interesting extension of this work).

Något för vidare studier vore att undersöka vad som händer om man slumpar starten.




% END 040.results.tex.
