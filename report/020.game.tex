% BEGIN 020.game.tex. Explanation of the game
\section{The game solution}\label{sec:game}
The game described in section \ref{sec:intro:game} is never relay simulated. Instead the game is seen as a Markov process, where the states are constituted of all possible histories of a given length. A transition matrix is then generated and the distribution of states is used to appreciate the average payoff for \pone\ and \ptwo. Consider the look-up table in figure \ref{fig:game:lookup}. There every row corresponds to a state in the Markov process, and e.i.\ \pdhifour{d}{d}{c}{d} denotes the history that both \pone\ and \ptwo\ chose \textsc{d} two rounds before and in the latest preciding round \pone\ chose \textsc{c} and \ptwo\ \textsc{d}.\mypar

The length of the table $N$ is decided by the longest memory of the two players according to
\begin{equation*}
N = 2\cdot\ceil{\frac{\max(M_\pone, M_\ptwo)}{2}}.
\end{equation*}
% Observe double usage of M here. This is a general problem in the presentation so far.


\begin{figure}[!htbp]\centering\footnotesize
\begin{tabular}{ccc}
Action & Hist & Action \\
\pone & & \ptwo \\
\hline
$s_{ 1} = p^{\textsc{i}}_{ 1}$ & \pdhifour{d}{d}{d}{d} & $t_{ 1} = p^{\textsc{ii}}_{ 1}$ \\
$s_{ 2} = p^{\textsc{i}}_{ 2}$ & \pdhifour{d}{d}{d}{c} & $t_{ 2} = p^{\textsc{ii}}_{ 3}$ \\
$s_{ 3} = p^{\textsc{i}}_{ 3}$ & \pdhifour{d}{d}{c}{d} & $t_{ 3} = p^{\textsc{ii}}_{ 2}$ \\
$s_{ 4} = p^{\textsc{i}}_{ 4}$ & \pdhifour{d}{d}{c}{c} & $t_{ 4} = p^{\textsc{ii}}_{ 4}$ \\ \hline
$s_{ 5} = p^{\textsc{i}}_{ 5}$ & \pdhifour{d}{c}{d}{d} & $t_{ 5} = p^{\textsc{ii}}_{ 9}$ \\
$s_{ 6} = p^{\textsc{i}}_{ 6}$ & \pdhifour{d}{c}{d}{c} & $t_{ 6} = p^{\textsc{ii}}_{11}$ \\
$s_{ 7} = p^{\textsc{i}}_{ 7}$ & \pdhifour{d}{c}{c}{d} & $t_{ 7} = p^{\textsc{ii}}_{10}$ \\
$s_{ 8} = p^{\textsc{i}}_{ 8}$ & \pdhifour{d}{c}{c}{c} & $t_{ 8} = p^{\textsc{ii}}_{12}$ \\ \hline
$s_{ 9} = p^{\textsc{i}}_{ 9}$ & \pdhifour{c}{d}{d}{d} & $t_{ 9} = p^{\textsc{ii}}_{ 5}$ \\
$s_{10} = p^{\textsc{i}}_{10}$ & \pdhifour{c}{d}{d}{c} & $t_{10} = p^{\textsc{ii}}_{ 7}$ \\
$s_{11} = p^{\textsc{i}}_{11}$ & \pdhifour{c}{d}{c}{d} & $t_{11} = p^{\textsc{ii}}_{ 6}$ \\
$s_{12} = p^{\textsc{i}}_{12}$ & \pdhifour{c}{d}{c}{c} & $t_{12} = p^{\textsc{ii}}_{ 8}$ \\ \hline
$s_{13} = p^{\textsc{i}}_{13}$ & \pdhifour{c}{c}{d}{d} & $t_{13} = p^{\textsc{ii}}_{13}$ \\
$s_{14} = p^{\textsc{i}}_{14}$ & \pdhifour{c}{c}{d}{c} & $t_{14} = p^{\textsc{ii}}_{15}$ \\
$s_{15} = p^{\textsc{i}}_{15}$ & \pdhifour{c}{c}{c}{d} & $t_{15} = p^{\textsc{ii}}_{14}$ \\
$s_{16} = p^{\textsc{i}}_{16}$ & \pdhifour{c}{c}{c}{c} & $t_{16} = p^{\textsc{ii}}_{16}$ \\ \hline
\end{tabular}

\caption{Look-up table for player tactics for players that at the most take four historical events into account.}
\label{fig:game:lookup}
\end{figure}

\subsection{The Transition Matrix}
The \textsc{pd} can be seen as the dynamic system
\begin{equation*}
h_{t+1} = M\, h_t
\end{equation*}
where $h_t$ is the state probability distribution at time $t$ and $M_{N\times N}= \{ m_{i,j}\}$ the transition matrix. The elements $m_{i, j}$ of $M$ thus corresponds to the probability for transition $j\rightarrow i$. The elements $m_{i, j}$ of $M$ are calculated in the following way.\mypar

Let $S = \myfrac{1}{2}\log_2\, N$ be the depth of the matrix with regard to blocks that divide the matrix in four. Use the measures
\begin{align*}
x &= (x_1, x_2, \dots, x_S) &\textrm{where} \quad x_s &= \floor{\frac{\makebox[1ex]{\textit{i}}-1}{4^{s-1}}} \mod 4 \quad \textrm{and}\\
y &= (y_1, y_2, \dots, y_S) &\textrm{where} \quad y_s &= \floor{\frac{\makebox[1ex]{\textit{j}}-1}{4^{s-1}}} \mod 4
\end{align*}
to get the values of the matrix elements according to
\begin{equation*}
m_{i, j} = \left\{
\begin{array}{ll}
\begin{array}{l}
A\,B\,(1-p_e)^2 + \\
(1 - A)\,B\, p_e(1-p_e) + \\
A\,(1-B)\,(1 - p_e)p_e + \\
(1-A)(1-B)\,p^2 \\[\baselineskip]
\end{array}
 & \textrm{if~}
\begin{array}{r@{~=~}l}
\quad y & (x_2, x_3, \dots, x_S, y_S) \textrm{~where~} \\
    y_S & 1, 2, 3, \textrm{~or~} 4
\end{array}
\\~\vspace{-0.5\baselineskip}~\\
\begin{array}{l}
0
\end{array}
 & \textrm{otherwise,}
\end{array}
\right.
\end{equation*}
and here
\begin{align*}
A &= \left\{%
\begin{array}{cl}%
(1-s_j) & \textrm{if~} x_1 = 1 \textrm{~or~} 2 \\
s_j & \textrm{otherwise}
\end{array}%
\right. \\[0.5\baselineskip]
B &= \left\{%
\begin{array}{cl}%
(1-t_j) & \textrm{if~} x_1 = 1 \textrm{~or~} 3 \\
t_j & \textrm{otherwise.}%
\end{array}%
\right.%
\end{align*}

The sought steady state distribution $h^\ast$ is achieved through diagonalisation of $M$. The distribution $h^\ast$ is the normalised eigenvector corresponding to the eigenvalue $\lambda^\ast = 1$.\mypar

The payoff $f$ for player \pone\ and \ptwo\ is then simply achieved in the following manner. First construct the state vector $g = (g_1, g_2, \dots, g_4)^\transp$ that corresponds to $M = 2$ %OBS
as 
\begin{displaymath}
g_l = \sum_{\mathrm{all~} k } h_i
\end{displaymath}
where $k$ are those $i$ for which $x_i = l$. The payoffs are the given as
\begin{align*}
f(\pone) &= 3\, g_4 + 5\, g_2 + g_1 \\
f(\ptwo) &= 3\, g_4 + 5\, g_3 + g_1.
\end{align*}

%------------------------------------------------------------------
\subsection{Strategy representation}

% END 020.game.tex
