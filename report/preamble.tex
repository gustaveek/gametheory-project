% Preamble to the game theory project.
% Gustav Eek 26 Feb 2010

\documentclass[a4paper, 11pt]{article}
\usepackage[english, swedish]{babel}
\usepackage[utf8]{inputenc}
\usepackage{a4wide}
\usepackage{graphicx}
\usepackage{latexsym, amssymb, amsmath}
%\usepackage{here}
%\usepackage{verbatim}
\usepackage{color}
\usepackage{enumerate}
\usepackage{verbatim}
\usepackage{epic, eepic}
\usepackage{url}
\usepackage{subfigure}
\usepackage[hang, footnotesize]{caption} % makes caption text align behind ``Figure 1''
\usepackage{multicol}
\usepackage{multirow}

% paragraph appearance
\parindent 0pt
\newcommand {\mypar} {\vspace{\baselineskip}\par}
\newcommand{\redCol}[1]{{\color{red} #1}}
\roman{subsubsection}

% lengths and spaces
\newlength{\myhspace}
\setlength{\myhspace}{1cm}
\newlength{\myvspace}
\setlength{\myvspace}{\baselineskip}
\newlength{\picwidth}
\setlength{\picwidth}{.40\textwidth}
\newcommand{\numberscale}{width=\picwidth, height=1.6\picwidth}
\newcommand{\matlabscale}{0.7}
\newcommand{\equspace}{\hspace{\myhspace}}
\newcommand{\arrayspace}{\vspace{\myvspace}}
\newcommand{\tablespace}{\vspace{.5\myvspace}}
\newcommand{\bildbredd}{.7\linewidth}

% semantic fount considerations
\newcommand \textco [1] {{\textit{#1}}}
\newcommand \myurl [1] {{\small\url{#1}}}
\newcommand \red [1] {{\color{red} #1}}

% text commands, abbreviations  and macros
\newcommand \pd {\textsc{pd}}
\newcommand \ipd {\textsc{ipd}}
\newcommand \pone {\textsc{pi}}  % player one
\newcommand \ptwo {\textsc{pii}} % player two
\newcommand \pdhitwo [2] {{% PD state with history two
\parbox[b]{1.17em}{\centering\footnotesize\textsc{#1#2}}%
}}
\newcommand \pdhifour [4] {{%
%\fbox{%
\parbox[c][1.8em][c]{1.17em}{\centering\footnotesize\textsc{#1#2}\vspace{-0.4\baselineskip}\\\textsc{#3#4}}%
%}%
}}
\newcommand \pdhisix [6] {{%
\fbox{%
\parbox[c][1.8em][c]{1.17em}{\centering\footnotesize\textsc{#1#2}\vspace{-0.5\baselineskip}\\\textsc{#3#4}\vspace{-0.5\baselineskip}\\\textsc{#5#6}}%
}%
}}
\newcommand \matlab {\textsc{Matlab}}

% math commands and macros
\newcommand{\dydt}{  \frac{dy}{dt}  }
\newcommand{\dudt}{  \frac{du}{dt}  }
\newcommand{\dydu}{  \frac{dy}{du}  }
\newcommand{\dfdx}{  \frac{df}{dx}  }
\newcommand{\dxdt}{  \frac{dx}{dt}  }
\newcommand \vardxdt {dx/dt}
\newcommand{\dkdt}{  \frac{dk}{dt}  }
\newcommand \vardkdt {dk/dt}
\newcommand{\dwdt}{  \frac{dw}{dt}  }
\newcommand \vardwdt {dw/dt}
\newcommand \barx {\bar{x}}
\renewcommand \epsilon \varepsilon
\newcommand{\myfrac}[2]{\hspace{1pt} ^{#1} \hspace{-0.8pt} / \hspace{-1.5pt} _{#2} \hspace{2pt} }
\newcommand{\ave}[1]{\left\langle #1 \right\rangle}
\newcommand{\avere}[2]{\ave{#1}_{(#2)}}
\newcommand{\norm}[1]{\left| #1 \right|}
\newcommand{\transp}{{\mathrm{T}}}
\newcommand{\tr}[1]{{#1}^\transp}
\newcommand{\sgn}[1]{{{\rm sgn}\left(#1 \right)}}
\newcommand{\fix}[2][] {{{#2}^\ast_{#1}}} % second optional argument.
\newcommand{\Exp} {\mathrm{Exp}}
\newcommand \ceil [1] {{\left\lceil{#1}\right\rceil}}
\newcommand \floor [1] {{\left\lfloor{#1}\right\rfloor}}


% enviornments
\newenvironment{code}{%
\begin{quote}
\begin{footnotesize}
\begin{ttfamily}
}%
{%
\end{ttfamily}
\end{footnotesize}
\end{quote}
}
