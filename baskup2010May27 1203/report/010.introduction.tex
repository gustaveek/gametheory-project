% BEGIN 010.introduction.tex
\section{Introduction}\label{sec:intro}
% brief history on prisoner's dilemma, iterated prisoners dilemma, iterated PD with genetic algorithms and noize, and also literature rewiew.
The Prisoner' Dilemma (\pd) concerns two accomplices in crime that face the decision of confession or silence in a prosecution situation. The puzzle, it is said, illustrates the conflict of individual and group rationality, and since the 50:s the game has been commonly used in game theory, \cite{stanford:pd}.\mypar

% iterated version

This report explains an experiment on iterated Prisoner's Dilemma (\pd) with mixed strategies and noise, where strategies are generated with genetic algorithms. This section and section and section \ref{sec:intro:game} gives a brief history and an introduction to the \pd-configuration used here. In section \ref{sec:game} the game model seen as a Markov process and the game solution is more deeply described, and in section \ref{sec:genetic} the exact model used as genetic strategy is explained. Finally in section \ref{sec:results} we give an account for simulation results, and a discussion is held.

%------------------------------------------------------------------
\subsection{The game}\label{sec:intro:game}
% explain with pure strategies
The game theoretical structure of the game is \dots \pone \ptwo
\begin{equation}
\begin{array}{cc c|c}
                     &            & \multicolumn{2}{c}{\pone} \\
                     &            & \textsc{c} & \textsc{d}   \\ \cline{3-4}
\multirow{2}*{\ptwo} & \textsc{c} & $(3, 3)$   & $(0, 5)$     \\ \cline{2-4}
		     & \textsc{d} & $(5, 0)$   & $(1, 1)$     \\ \cline{3-4}
\end{array}
\label{equ:intro:payoff}
\end{equation}
g
% iterated game 


% explain with mixed strategies


% the noise



% END 010.introduction.tex.
